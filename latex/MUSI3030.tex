% Generated by Sphinx.
\def\sphinxdocclass{report}
\documentclass[letterpaper,10pt,english]{sphinxmanual}
\usepackage[utf8]{inputenc}
\DeclareUnicodeCharacter{00A0}{\nobreakspace}
\usepackage[T1]{fontenc}
\usepackage{babel}
\usepackage{times}
\usepackage[Bjarne]{fncychap}
\usepackage{longtable}
\usepackage{sphinx}
\usepackage{multirow}


\title{MUSI3030 Documentation}
\date{August 22, 2013}
\release{0}
\author{Daniel Shanahan}
\newcommand{\sphinxlogo}{}
\renewcommand{\releasename}{Release}
\makeindex

\makeatletter
\def\PYG@reset{\let\PYG@it=\relax \let\PYG@bf=\relax%
    \let\PYG@ul=\relax \let\PYG@tc=\relax%
    \let\PYG@bc=\relax \let\PYG@ff=\relax}
\def\PYG@tok#1{\csname PYG@tok@#1\endcsname}
\def\PYG@toks#1+{\ifx\relax#1\empty\else%
    \PYG@tok{#1}\expandafter\PYG@toks\fi}
\def\PYG@do#1{\PYG@bc{\PYG@tc{\PYG@ul{%
    \PYG@it{\PYG@bf{\PYG@ff{#1}}}}}}}
\def\PYG#1#2{\PYG@reset\PYG@toks#1+\relax+\PYG@do{#2}}

\expandafter\def\csname PYG@tok@gd\endcsname{\def\PYG@tc##1{\textcolor[rgb]{0.63,0.00,0.00}{##1}}}
\expandafter\def\csname PYG@tok@gu\endcsname{\let\PYG@bf=\textbf\def\PYG@tc##1{\textcolor[rgb]{0.50,0.00,0.50}{##1}}}
\expandafter\def\csname PYG@tok@gt\endcsname{\def\PYG@tc##1{\textcolor[rgb]{0.00,0.27,0.87}{##1}}}
\expandafter\def\csname PYG@tok@gs\endcsname{\let\PYG@bf=\textbf}
\expandafter\def\csname PYG@tok@gr\endcsname{\def\PYG@tc##1{\textcolor[rgb]{1.00,0.00,0.00}{##1}}}
\expandafter\def\csname PYG@tok@cm\endcsname{\let\PYG@it=\textit\def\PYG@tc##1{\textcolor[rgb]{0.25,0.50,0.56}{##1}}}
\expandafter\def\csname PYG@tok@vg\endcsname{\def\PYG@tc##1{\textcolor[rgb]{0.73,0.38,0.84}{##1}}}
\expandafter\def\csname PYG@tok@m\endcsname{\def\PYG@tc##1{\textcolor[rgb]{0.13,0.50,0.31}{##1}}}
\expandafter\def\csname PYG@tok@mh\endcsname{\def\PYG@tc##1{\textcolor[rgb]{0.13,0.50,0.31}{##1}}}
\expandafter\def\csname PYG@tok@cs\endcsname{\def\PYG@tc##1{\textcolor[rgb]{0.25,0.50,0.56}{##1}}\def\PYG@bc##1{\setlength{\fboxsep}{0pt}\colorbox[rgb]{1.00,0.94,0.94}{\strut ##1}}}
\expandafter\def\csname PYG@tok@ge\endcsname{\let\PYG@it=\textit}
\expandafter\def\csname PYG@tok@vc\endcsname{\def\PYG@tc##1{\textcolor[rgb]{0.73,0.38,0.84}{##1}}}
\expandafter\def\csname PYG@tok@il\endcsname{\def\PYG@tc##1{\textcolor[rgb]{0.13,0.50,0.31}{##1}}}
\expandafter\def\csname PYG@tok@go\endcsname{\def\PYG@tc##1{\textcolor[rgb]{0.20,0.20,0.20}{##1}}}
\expandafter\def\csname PYG@tok@cp\endcsname{\def\PYG@tc##1{\textcolor[rgb]{0.00,0.44,0.13}{##1}}}
\expandafter\def\csname PYG@tok@gi\endcsname{\def\PYG@tc##1{\textcolor[rgb]{0.00,0.63,0.00}{##1}}}
\expandafter\def\csname PYG@tok@gh\endcsname{\let\PYG@bf=\textbf\def\PYG@tc##1{\textcolor[rgb]{0.00,0.00,0.50}{##1}}}
\expandafter\def\csname PYG@tok@ni\endcsname{\let\PYG@bf=\textbf\def\PYG@tc##1{\textcolor[rgb]{0.84,0.33,0.22}{##1}}}
\expandafter\def\csname PYG@tok@nl\endcsname{\let\PYG@bf=\textbf\def\PYG@tc##1{\textcolor[rgb]{0.00,0.13,0.44}{##1}}}
\expandafter\def\csname PYG@tok@nn\endcsname{\let\PYG@bf=\textbf\def\PYG@tc##1{\textcolor[rgb]{0.05,0.52,0.71}{##1}}}
\expandafter\def\csname PYG@tok@no\endcsname{\def\PYG@tc##1{\textcolor[rgb]{0.38,0.68,0.84}{##1}}}
\expandafter\def\csname PYG@tok@na\endcsname{\def\PYG@tc##1{\textcolor[rgb]{0.25,0.44,0.63}{##1}}}
\expandafter\def\csname PYG@tok@nb\endcsname{\def\PYG@tc##1{\textcolor[rgb]{0.00,0.44,0.13}{##1}}}
\expandafter\def\csname PYG@tok@nc\endcsname{\let\PYG@bf=\textbf\def\PYG@tc##1{\textcolor[rgb]{0.05,0.52,0.71}{##1}}}
\expandafter\def\csname PYG@tok@nd\endcsname{\let\PYG@bf=\textbf\def\PYG@tc##1{\textcolor[rgb]{0.33,0.33,0.33}{##1}}}
\expandafter\def\csname PYG@tok@ne\endcsname{\def\PYG@tc##1{\textcolor[rgb]{0.00,0.44,0.13}{##1}}}
\expandafter\def\csname PYG@tok@nf\endcsname{\def\PYG@tc##1{\textcolor[rgb]{0.02,0.16,0.49}{##1}}}
\expandafter\def\csname PYG@tok@si\endcsname{\let\PYG@it=\textit\def\PYG@tc##1{\textcolor[rgb]{0.44,0.63,0.82}{##1}}}
\expandafter\def\csname PYG@tok@s2\endcsname{\def\PYG@tc##1{\textcolor[rgb]{0.25,0.44,0.63}{##1}}}
\expandafter\def\csname PYG@tok@vi\endcsname{\def\PYG@tc##1{\textcolor[rgb]{0.73,0.38,0.84}{##1}}}
\expandafter\def\csname PYG@tok@nt\endcsname{\let\PYG@bf=\textbf\def\PYG@tc##1{\textcolor[rgb]{0.02,0.16,0.45}{##1}}}
\expandafter\def\csname PYG@tok@nv\endcsname{\def\PYG@tc##1{\textcolor[rgb]{0.73,0.38,0.84}{##1}}}
\expandafter\def\csname PYG@tok@s1\endcsname{\def\PYG@tc##1{\textcolor[rgb]{0.25,0.44,0.63}{##1}}}
\expandafter\def\csname PYG@tok@gp\endcsname{\let\PYG@bf=\textbf\def\PYG@tc##1{\textcolor[rgb]{0.78,0.36,0.04}{##1}}}
\expandafter\def\csname PYG@tok@sh\endcsname{\def\PYG@tc##1{\textcolor[rgb]{0.25,0.44,0.63}{##1}}}
\expandafter\def\csname PYG@tok@ow\endcsname{\let\PYG@bf=\textbf\def\PYG@tc##1{\textcolor[rgb]{0.00,0.44,0.13}{##1}}}
\expandafter\def\csname PYG@tok@sx\endcsname{\def\PYG@tc##1{\textcolor[rgb]{0.78,0.36,0.04}{##1}}}
\expandafter\def\csname PYG@tok@bp\endcsname{\def\PYG@tc##1{\textcolor[rgb]{0.00,0.44,0.13}{##1}}}
\expandafter\def\csname PYG@tok@c1\endcsname{\let\PYG@it=\textit\def\PYG@tc##1{\textcolor[rgb]{0.25,0.50,0.56}{##1}}}
\expandafter\def\csname PYG@tok@kc\endcsname{\let\PYG@bf=\textbf\def\PYG@tc##1{\textcolor[rgb]{0.00,0.44,0.13}{##1}}}
\expandafter\def\csname PYG@tok@c\endcsname{\let\PYG@it=\textit\def\PYG@tc##1{\textcolor[rgb]{0.25,0.50,0.56}{##1}}}
\expandafter\def\csname PYG@tok@mf\endcsname{\def\PYG@tc##1{\textcolor[rgb]{0.13,0.50,0.31}{##1}}}
\expandafter\def\csname PYG@tok@err\endcsname{\def\PYG@bc##1{\setlength{\fboxsep}{0pt}\fcolorbox[rgb]{1.00,0.00,0.00}{1,1,1}{\strut ##1}}}
\expandafter\def\csname PYG@tok@kd\endcsname{\let\PYG@bf=\textbf\def\PYG@tc##1{\textcolor[rgb]{0.00,0.44,0.13}{##1}}}
\expandafter\def\csname PYG@tok@ss\endcsname{\def\PYG@tc##1{\textcolor[rgb]{0.32,0.47,0.09}{##1}}}
\expandafter\def\csname PYG@tok@sr\endcsname{\def\PYG@tc##1{\textcolor[rgb]{0.14,0.33,0.53}{##1}}}
\expandafter\def\csname PYG@tok@mo\endcsname{\def\PYG@tc##1{\textcolor[rgb]{0.13,0.50,0.31}{##1}}}
\expandafter\def\csname PYG@tok@mi\endcsname{\def\PYG@tc##1{\textcolor[rgb]{0.13,0.50,0.31}{##1}}}
\expandafter\def\csname PYG@tok@kn\endcsname{\let\PYG@bf=\textbf\def\PYG@tc##1{\textcolor[rgb]{0.00,0.44,0.13}{##1}}}
\expandafter\def\csname PYG@tok@o\endcsname{\def\PYG@tc##1{\textcolor[rgb]{0.40,0.40,0.40}{##1}}}
\expandafter\def\csname PYG@tok@kr\endcsname{\let\PYG@bf=\textbf\def\PYG@tc##1{\textcolor[rgb]{0.00,0.44,0.13}{##1}}}
\expandafter\def\csname PYG@tok@s\endcsname{\def\PYG@tc##1{\textcolor[rgb]{0.25,0.44,0.63}{##1}}}
\expandafter\def\csname PYG@tok@kp\endcsname{\def\PYG@tc##1{\textcolor[rgb]{0.00,0.44,0.13}{##1}}}
\expandafter\def\csname PYG@tok@w\endcsname{\def\PYG@tc##1{\textcolor[rgb]{0.73,0.73,0.73}{##1}}}
\expandafter\def\csname PYG@tok@kt\endcsname{\def\PYG@tc##1{\textcolor[rgb]{0.56,0.13,0.00}{##1}}}
\expandafter\def\csname PYG@tok@sc\endcsname{\def\PYG@tc##1{\textcolor[rgb]{0.25,0.44,0.63}{##1}}}
\expandafter\def\csname PYG@tok@sb\endcsname{\def\PYG@tc##1{\textcolor[rgb]{0.25,0.44,0.63}{##1}}}
\expandafter\def\csname PYG@tok@k\endcsname{\let\PYG@bf=\textbf\def\PYG@tc##1{\textcolor[rgb]{0.00,0.44,0.13}{##1}}}
\expandafter\def\csname PYG@tok@se\endcsname{\let\PYG@bf=\textbf\def\PYG@tc##1{\textcolor[rgb]{0.25,0.44,0.63}{##1}}}
\expandafter\def\csname PYG@tok@sd\endcsname{\let\PYG@it=\textit\def\PYG@tc##1{\textcolor[rgb]{0.25,0.44,0.63}{##1}}}

\def\PYGZbs{\char`\\}
\def\PYGZus{\char`\_}
\def\PYGZob{\char`\{}
\def\PYGZcb{\char`\}}
\def\PYGZca{\char`\^}
\def\PYGZam{\char`\&}
\def\PYGZlt{\char`\<}
\def\PYGZgt{\char`\>}
\def\PYGZsh{\char`\#}
\def\PYGZpc{\char`\%}
\def\PYGZdl{\char`\$}
\def\PYGZhy{\char`\-}
\def\PYGZsq{\char`\'}
\def\PYGZdq{\char`\"}
\def\PYGZti{\char`\~}
% for compatibility with earlier versions
\def\PYGZat{@}
\def\PYGZlb{[}
\def\PYGZrb{]}
\makeatother

\begin{document}

\maketitle
\tableofcontents
\phantomsection\label{index::doc}


Here you can find the course syllabus, assignments, lesson plans, supplemental materials, and useful links. Please feel free to contact me if you have any questions or comments.


\chapter{Syllabus}
\label{syllabus:welcome-to-class-website-for-musi3030-nineteenth-century-music}\label{syllabus:syllabus}\label{syllabus::doc}

\section{Studies in Nineteenth Century Music (MUSI3030)}
\label{syllabus:studies-in-nineteenth-century-music-musi3030}\begin{quote}\begin{description}
\item[{Instructor}] \leavevmode
Dr. Daniel Shanahan

\item[{Office}] \leavevmode
Room 129 (basement), Kerchof Hall (Math Building; on C parking lot across from Clark Hall)

\item[{E-mail}] \leavevmode
\href{mailto:dts9h@virginia.edu}{dts9h@virginia.edu}

\item[{Office hours}] \leavevmode
MW 2-3pm; R 2:30-4:30 and By Appointment

\item[{Course}] \leavevmode
MUSI3030

\item[{Credit}] \leavevmode
3.0 credits

\item[{Term}] \leavevmode
Fall 2013

\item[{Time}] \leavevmode
TR / 12:30-1:45pm

\item[{Place}] \leavevmode
Old Cabel Hall, B012

\item[{Class Number}] \leavevmode
16336

\item[{Text}] \leavevmode
Richard Taruskin, \emph{Music in the Nineteenth Century} (Oxford, 2009; paperback, ISBN 0195384830)

\item[{Course Web Page}] \leavevmode
\href{http://shanahdt.github.io/MUSI3030/}{http://shanahdt.github.io/MUSI3030/}

\end{description}\end{quote}


\subsection{Course Objectives}
\label{syllabus:course-objectives}
This course surveys European music in the nineteenth century.
We will cover a wide array of composers (Chopin, Liszt, Schumann, Berlioz,
Wagner, and Verdi, among others), genres (solo instrumental, art song, choral,
instrumental chamber music, symphony, opera), nations and regions (France,
Germany and Austria, Italy, Russia, North America), and topics (salon culture, virtuosity,
folk music, orientalism, musical meaning, etc.).
\textbar{}
We will be reading scores throughout the semester. Theory 1 (MUSI 3310) strongly recommended as a prerequisite.
\textbar{}
For music majors: This course fulfills either the
first historical requirement or an elective requirement.
\textbar{}
Enrollment deadlines: The last day in the College to add a course is Tuesday,
September 10th; the last day to drop a course is Wednesday, September 11th;
the last day to withdraw from a course is Tuesday, October 22nd.


\subsection{Resources}
\label{syllabus:resources}
The following texts are required and are available either at the UVa Bookstore or through online sellers.


\subsubsection{Required Text}
\label{syllabus:required-text}
Richard Taruskin, \emph{Music in the Nineteenth Century} (Oxford, 2009; paperback, ISBN 0195384830)


\subsubsection{Suggested Reading}
\label{syllabus:suggested-reading}
Carl Dahlhaus, \emph{Nineteenth-Century Music} (University of California Press, 1991; paperback, ISBN 0520076443)

Piero Weiss and Richard Taruskin, \emph{Music in the Western World} (Schirmer, 2007; paperback, 2nd edition, ISBN 053458599X)

Note that while I encourage building a library of such important
musicological texts, these books can be quite expensive. I will work
to put as much of these suggested readings on either this website or
Collab, along with musical scores. For the listening assignments, I will
direct you either to a streaming database to which the UVa library subscribes,
or good quality recordings on YouTube. Links will also be
made available on the course website.

\textbf{NOTE:} If you have difficulty locating material for this course, please let me know immediately by email.


\subsection{Website}
\label{syllabus:website}
The course website can be found at
\href{http://shanahdt.github.io/MUSI3030/}{http://shanahdt.github.io/MUSI3030/}. This site contains lectures,
course materials, supplementary readings, quizzes and
assignments \textbf{for self assessment}, and helpful links.
It is intended to complement, rather than replace, Collab.


\subsection{Assistance}
\label{syllabus:assistance}
I am available and interested in talking with you about the course,
the course material, and strategies to enhance your learning.
We can usually have brief discussions after class, and I am able
to answer questions by e-mail (\href{mailto:dts9h@virginia.edu}{dts9h@virginia.edu}) at any time.
Additionally, I am happy to set up an appointment at a time that
is mutually acceptable for more lengthy discussions.


\subsection{Grading and Class Activities}
\label{syllabus:grading-and-class-activities}
\begin{tabulary}{\linewidth}{|L|L|}
\hline

Attendance and Participation
 & 
20\%
\\\hline

Weekly Question Sheets
 & 
10\%
\\\hline

Biweekly Listening Comparisons
 & 
10\%
\\\hline

Quiz 1
 & 
15\%
\\\hline

Quiz 2
 & 
15\%
\\\hline

In-Class Presentation
 & 
5\%
\\\hline

Final Paper Outline and Bibliography
 & 
5\%
\\\hline

Final Paper
 & 
5\%
\\\hline
\end{tabulary}



\subsection{Attendance and Participation}
\label{syllabus:attendance-and-participation}
You are expected to attend every class. After 2 absences
I start to lower your grade. (I do not distinguish between
excused and unexcused absences, except in extreme instances
and supported by appropriate documentation.) Your participation
grade is dependent upon your contribution to the class discussion.
To earn maximum points for class discussion, try to contribute regularly and thoughtfully.


\subsection{Weekly Question Sheets}
\label{syllabus:weekly-question-sheets}
There will be questions on the readings due every Tuesday.
Answer these questions as succinctly as possible--usually one or
two words per answer will suffice. Please submit these in \textbf{hard copy, pledged}
(See \emph{Honor} section below).


\subsection{Biweekly Listening Comparisons}
\label{syllabus:biweekly-listening-comparisons}
Every other Thursday you will be asked to write 1-2 pages
(about 500 words, double-spaced in 12-point font, with page numbers)
in which you compare two or three similar pieces assigned for that day.
I will provide an example of such a comparison as a model for your own essays.
Please submit these in hard copy, pledged. Since this assignment is biweekly,
the class will be divided into two halves--Groups A (even weeks)
and B (odd weeks)--and you will submit your assignment according to your
grouping. Since the odd weeks outnumber the even weeks by
one on our schedule, the members of the odd-week group
(Group B) do not have to submit a comparison for one week of their choice.


\subsection{Quizzes}
\label{syllabus:quizzes}

\subsubsection{Quiz 1}
\label{syllabus:quiz-1}
Placed at about a third of the way through the course,
Quiz 1 will comprise mainly short-answer questions selected
from the Question Sheets and listening identifications from the
assigned listenings. This Quiz will be administered during class.


\subsubsection{Quiz 2}
\label{syllabus:quiz-2}
This Quiz will have the same design as Quiz 1, but
it will occur at about two thirds of the way through the course.


\subsection{In-Class Presentation}
\label{syllabus:in-class-presentation}
You’ll pair up with another student to orally present your
Listening Comparison on a Thursday during the semester.
Since you'll be presenting together, you may submit either
one written comparison for the two of you or two separate comparisons.
We will be assigning days to presenters on the second day of class
(Thursday, August 29th), so if you have a preference for either
a partner or a particular day or both, please come to class
with your preferences in mind. I cannot guarantee that you
will get your first choice, but you should feel assured that
all the listening assignments are equally good. Please
include audio-visual elements in your presentation: listening
excerpts, handouts, slideshow, etc. When listening to
presentations by other students, be respectful
and attentive, take notes, and be prepared to ask
follow-up questions. I am expecting that those who belong
to the same Comparison Group will be especially lively
contributors to discussion, since they'll just have
completed the same exercise.


\subsection{Final Paper}
\label{syllabus:final-paper}
Your final, 8- to 10-page paper will be due at 5pm on Monday, December, 9th.
I will meet with each of you individually during the week of Monday, November 11th,
to discuss possible paper topics. Sign-up will be on Collab the previous week.
Please come to our session with at least two possibilities in mind. You are
also required to submit a 1-page, single-spaced outline and a 1-page,
single-spaced bibliography by the beginning of class on November 21st.
I expect to find at least 6 distinct and substantial sources in your
bibliography. (A wikipedia article is not a substantial source, nor is a blog.)
Be succinct in your outline and judicious in your choice of sources
for your bibliography. Since research methods and materials are specific
to the topic you choose, we will strategize about them
during our individual meetings in early November.


\subsection{Recommendation for Listening}
\label{syllabus:recommendation-for-listening}
Try to listen more than once to a piece. Do not let yourself be
distracted while doing so. If you listen on a computer, try to
use good quality headphones, rather than using flimsy ear
buds or just letting the music play through your computer
speakers. I also recommend always following along with
either a score or a libretto (when applicable).


\subsection{Policy on Late Assignments}
\label{syllabus:policy-on-late-assignments}
Since we will be discussing the answers immediately in
class, Question Sheets and Biweekly Comparisons will not
be accepted for a grade if they are submitted after the
beginning of the class in which they're due. For the final
paper and its preparation (the outline and bibliography),
the grade will be lowered one increment (for example, B+ to B)
for every day they are late.


\subsection{Lecture and Reading Schedule}
\label{syllabus:lecture-and-reading-schedule}

\subsubsection{Unit I}
\label{syllabus:unit-i}
\begin{tabular}{|p{0.317\linewidth}|p{0.317\linewidth}|p{0.317\linewidth}|}
\hline
\textbf{
Week
} & \textbf{
Topic
} & \textbf{
Assignments and Readings Due
}\\\hline

1.T
(8/27)
 & 
Introduction
 & \\\hline

1.R
(8/29)
 & 
Beethoven
 & 
Sign up for Listening Comparison Presentations.

\textbf{Due}:Read Taruskin, ``The First Romantics'' on Collab.
Listen to Beethoven, Symphony No.3 in E-flat major (``Eroica'')
Answer Question Sheet No. 1
\\\hline

2.T
(9/3)
 & 
Beethoven,
\emph{continued}
 & 
Read Hoffman, ``Beethoven's Instrumental Music'' (on Collab).
\textbf{Due}: Listen comparatively to Beethoven's op.13 (``Pathetique'') and
op.111. Feel free to compare one entire sonata to the other, or
corresponding movements to one another.

\textbf{Group A} Listening Comparison Due
\\\hline

2.R
(9/5)
 & 
Rossini
 & 
Read Taruskin, pp.1-36
Listen to the Overture and Act I from Beethoven's \emph{Fidelio}

Answer Question Sheet No.2
\\\hline

3.T
(9/10)
 & 
Rossini,
\emph{continued}
 & 
Listen comparatively to the Overture and Act I from Mozart's \emph{Le
nozze di Figaro} and the Overture and Act I from Rossini's \emph{Il
barbiere di Siviglia}

\textbf{Group B} Listening Comparison Due
\\\hline

3.R
(9/12)
 & 
Schubert
 & 
Read Taruskin, pp.61-87 and pp.135-155

Listen to three pieces by Franz Schubert:
\begin{itemize}
\item {} 
``Gretchen am Spinnrade'' (song)

\item {} 
``Erlkonig''(song)

\item {} 
\emph{String Quartet in C major}

\end{itemize}
\\\hline

4.T
(9/17)
 & 
Schubert,
\emph{continued}
 & 
Listen comparatively to music settings of Goethe's ``Erlkonig''
by Schubert, Zelter, and Reichardt.

Listen to Schubert's \emph{Die schone Mullerin} (song cycle)
\textbf{Group A} Listening Comparison Due
\\\hline

4.R

(9/19)
 & 
Grand and Gothic
Opera
 & 
Read Taruskin, pp.187-205 and 219-230

Listen to:
\begin{itemize}
\item {} 
Weber, Overture to \emph{Der Freischutz}

\item {} 
Meyerbeer, Act IV, \emph{Les Huguenots}

\end{itemize}

Answer Question Sheet 4
\\\hline

5.T
(9/24)
 & 
Grand and Gothic
Opera, \emph{continued}
 & 
Read Taruskin, pp.179-186
Listen comparatively to:

-Mendelssohn, Overture to \emph{A Midsummer Night's Dream}
-Weber, Overture to \emph{Oberon}

\textbf{Group B} Listening Comparison Due
\\\hline

5.R
(9/26)
 & 
Virtuosity
 & 
Quiz 1
\\\hline

6.T
(10/1)
 & 
Virtuosity,
\emph{continued}
 & 
Read Weiss/Taruskin, pp.289-295 and pp.308-313 (on Collab)

Listen comparatively to three versions of Liszt's study in
C minor, first composed in 1826 (from his \emph{Etude en 12} and
revised in both 1837 (\emph{12 grandes etudes}) and 1851 (\emph{Etudes
d'execution transcendante}

Listen to a selection of studies by Paganini (for violin) and Liszt
(for piano). Selections are on Collab.

\textbf{Group A} Listening Comparison Due
\\\hline

6.R
(10/3)
 & 
Schumann and Berlioz
 & 
Read Taruskin, 289-341
Listen to:
\begin{itemize}
\item {} 
Schumann, \emph{Papillons}

\item {} 
Berlioz, \emph{Symphonie fantastique}

\end{itemize}

Answer Question Sheet 5
\\\hline

7.T
(10/8)
 & 
Schumann and Berlioz,
\emph{continued}
 & 
Read Weiss/Taruskin, 296-300 and 303-308 (on Collab)
Listen to:
\begin{itemize}
\item {} 
Berlioz, \emph{Harold en Italie} and compare the musical devices used
and Lord Byron's text setting (available on Collab).

\end{itemize}

\textbf{Group B} Listening Comparison Due
\\\hline

7.R
(10/10)
 & 
Chopin, Gottschalk,
and Orientalism
 & 
Read Taruskin, pp.343-386
Listen to:
\begin{itemize}
\item {} 
Chopin, \emph{Preludes}

\item {} 
Chopin, \emph{Four Mazurkas} (op.17)

\item {} 
Chopin, \emph{Ballade no.1 in G minor}

\item {} 
Gottschalk, Bamboula

\end{itemize}

Answer Question Sheet 6
\\\hline

8.T
(10/15)
 & 
Reading Day
 & \\\hline

8.R
(10/17)
 & 
Chopin, Gottschalk,
and Orientalism,
\emph{continued}
 & 
Read Taruskin, pp.386-410
Listen to:
\begin{itemize}
\item {} 
Borodin, Polovtsian Dances from \emph{Prince Igor}

\item {} 
Cui, \emph{The Mandarin's Son}

\end{itemize}

\textbf{Group A} Listening Comparison Due
\\\hline

9.T
(10/22)
 & 
Liszt
 & 
Read Taruskin, pp.411-428
Listen to:
\begin{itemize}
\item {} 
Liszt, \emph{Les Preludes}

\item {} 
Liszt, \emph{A Faust Symphony}

\end{itemize}

Answer Question Sheet 7
\\\hline

9.R
(10/24)
 & 
Liszt, \emph{continued}
 & 
Read Taruskin, pp.438-442
Read Weiss/Taruskin, pp.324-329 (on Collab)

Listen comparatively to any two of Liszt's 19 Hungarian Rhapsodies

\textbf{Group B} Listening Comparison Due
\\\hline

10.T
(10/29)
 & 
Dvorak and Smetana
 & 
Read Taruskin, 443-463
Listen to:
\begin{itemize}
\item {} 
Smetana, \emph{Libuse}, Act 1

\item {} 
Dvorak, \emph{Rusalka}, Act 1

\end{itemize}

Answer Question Sheet 7
\\\hline

10.R
(10/31)
 & 
Dvorak and Smetana,
\emph{continued}
 & 
Halloween (Topical Costumes Encouraged)

Listen comparatively to:
\begin{itemize}
\item {} 
Smetana, Vltava, from \emph{M'a Vlast}

\item {} 
Smetana, Blanik, from \emph{M'a Vlast}

\item {} 
Dvorak, Allegro ma non Troppo, \emph{String Quartet no.12} (American)

\item {} 
Dvorak, Lento, \emph{String Quartet no.12} (American)

\end{itemize}

\textbf{Group A} Listening Comparison Due
\\\hline

11.T
(11/5)
 & 
Wagner I
 & 
Read Taruskin, pp.479-520

Listen to the instrumental preludes to three Wagner Operas:
\begin{itemize}
\item {} 
\emph{The Flying Dutchman}

\item {} 
\emph{Tannhauser}

\item {} 
\emph{Tristan und Isolde}

\end{itemize}

Answer Question Sheet 8
\\\hline

11.R
(11/7)
 & 
Wagner I, \emph{continued}
 & 
Read Taruskin, pp.528-562

Listen comparatively to:
\begin{itemize}
\item {} 
Prelude to \emph{Lohengrin}

\item {} 
Prelude to \emph{Parsifal}

\end{itemize}

\textbf{Group B} Listening Comparison Due
\\\hline

12.T
(11/12)
 & 
Wagner II
 & 
Quiz 2
\\\hline

12.R
(11/14)
 & 
Wagner II, \emph{continued}
 & 
Read Wagner, ``The Artwork of the Future'' (on Collab)

Listen comparatively to two love duets:
\begin{itemize}
\item {} 
Wagner, ``O sink hernieder, Nacht der Liebe,'' Act II, \emph{Tristan}

\item {} 
``Gia nella notte densa,'' from Act I, \emph{Otello}

\end{itemize}

\textbf{Group A} Listening Comparison Due
(How does each composer engage both singers and orchestra to represent
nighttime ecstasy?)
\\\hline

13.T
(11/19)
 & 
Italian Opera
 & 
Read Taruskin, pp.564-615
Listen to/watch \emph{La Traviata}

Answer Question Sheet 9
\\\hline

13.R

(11/21)
 & 
Italian Opera \emph{cont.}
 & 
Final Paper Outline and Bibliography Due
Read Taruskin, pp.639-658, pp.658-674

Listen comparatively to two ``mad scenes'':
\begin{itemize}
\item {} 
``Una macchia e qui tutt'ora'', Lady MacBeth in Verdi's \emph{Macbeth}

\item {} \begin{description}
\item[{“Il dolce suono...Spargi d'amaro pianto”, Lucia from Donizetti's}] \leavevmode
\emph{Lucia di Lammermoor}

\end{description}

\end{itemize}

\textbf{Group B} Listening Comparison Due
(How does each composer depict psychological unrest in the
vocal and instrumental writing?)
\\\hline

14.T
(11/26)
 & 
Brahms
 & 
Read Taruskin, pp.675-702 and pp.716-729

Listen to Brahms, \emph{Symphony No.1}
Answer Question Sheet 10
\\\hline

14.R
(11/28)
 & 
Brahms, \emph{continued}
 & 
Read Eduard Hanslick, ``On the Musically Beautiful'' (excerpt)

Listen comparatively to two Brahms Intermezzi:
\begin{itemize}
\item {} 
Op. 118, No.2

\item {} 
Op. 118, No.6

\end{itemize}
\\\hline

15.T
(12/3)
 & 
Final Week
 & 
Final Class and Party

\textbf{Your final, 8- to 10-page paper is due on Monday, December 9th.}
\\\hline
\end{tabular}



\subsection{Honor}
\label{syllabus:honor}
I trust every student in this course to comply with all of the provisions of the UVA honor system.
I will ask that you pledge and sign the two examinations and three papers.
Your signature on the exams affirms you have not received nor given aid while
taking your exams, nor accessed any notes, study outlines, old exams, answer keys,
or books  while taking an exam and that you have not obtained any answers from another
student's exam.  Your signature on the papers affirms that they represent your original
work, and that any sources you have quoted, paraphrased, or used extensively in preparing
the paper have been properly credited in the footnotes or bibliography.


\subsection{Students with disabilities}
\label{syllabus:students-with-disabilities}
This syllabus is available in alternative formats (PDF, HTML, epub) upon
request. In addition, if you may need an accommodation based on the
impact of a disability, you should contact me immediately.
Students with special needs can contact UVa's Office of Disability
Support Services (ph: 276-328-0265, email: \href{mailto:wew3x@uvawise.edu}{wew3x@uvawise.edu}) with any questions.
I will make every effort to accommodate special needs.



\renewcommand{\indexname}{Index}
\printindex
\end{document}
